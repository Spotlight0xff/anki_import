\documentclass[a4paper,12pt]{article}
\usepackage{enumerate}
%\usepackage{ngerman}
%\usepackage[latin1]{inputenc}
\usepackage{t1enc} % notwendig für hyphenation
\usepackage[labelfont=bf]{caption} % Captions fett
\usepackage[ngerman]{babel}
\usepackage[utf8]{inputenc}
\usepackage[babel,german=quotes]{csquotes}
\usepackage{amsmath}
\usepackage{amssymb}
\usepackage{subfigure} % mehrere Grafiken in einer Figure-Umgebung
\usepackage{texdraw}
\usepackage{graphicx}
\usepackage{enumitem}
\usepackage{color}
\setlist{nolistsep}
%\usepackage{url}
\usepackage[hidelinks]{hyperref}
\usepackage{siunitx}
\sisetup{locale=DE} % Komma als Dezimaltrennzeichen, global definiert
\hyphenation{Nah-kan-ten Dun-kel-zähl-rate Ge-mein-schaft Mess-sys-te-me Ep-oxid-har-ze Teub-ner win-kel-auf-ge-lös-ter element-spe-zi-fische element-spe-zi-fisch Leck-such-test Ener-gie-po-si-tion äqui-va-len-ten}
\usepackage[bottom]{footmisc}
\newcommand{\versuchname}{Hunklinger - Festkörperphysik}
\usepackage{scrpage2}

\pagestyle{scrheadings}
\ihead{\versuchname}
\ohead{\thepage}
\setheadsepline{0.4pt}

\setlength{\textwidth}{14cm}
\setlength{\textheight}{22cm}

\renewcommand\thefootnote{[\arabic{footnote}]}

\newcommand{\eq[1]}{\begin{equation}{#1}\end{equation}}

%\renewcommand{\figurename}{\textit{Abb.}}
%\renewcommand{\tablename}{\textit{Tab.}}

\title{\vspace{-2cm}
	Praktikum Moderne Physik\\
	~\\
	Auswertung:\\
	\textbf{\versuchname}
}

\author{
	Gruppe Mo-103 (SS15): Jasmin Bouazza, Marco Merboldt\\
}

%\date{\today}
\date{26. Oktober 2015}

% Anführungszeichen
%"`Beispiel"' =  (" + `) {blubb} (" + ')

% Aufzählungspunkte
%\begin{itemize}
%	\item \dots
%\end{itemize}

% Alphabetische Nummerierung
%\begin{enumerate}[label=\textbf{\alph*)}]
%	\item \dots
%	\item \dots
%\end{enumerate}

% Grafik einbinden
%\begin{figure}[h]
%	\centering
%	\includegraphics[width=\textwidth]{path/name.png}
%	\caption{description}
%	\label{ref}
%\end{figure}

% Tabelle einfügen
%\begin{table}
%	\caption{Tabellenüberschrift}
%	\label{tab:ref}
%	\centering
%	\begin{tabular}{|c|rl|}\hline
%		\textbf{c1} 	& \textbf{c1}	& \textbf{c1} 	\\ \hline\hline
%		r1c1 		& r1c2			& r1c3 		\\ \hline
%		r2c1 		& r2c2			& r2c3 		\\ \hline
%	\end{tabular}
%\end{table}

% Fußntoe
%\footnote{z.B. Anmerkung, Quelle, ...}

% Gleichungen
%\begin{equation}
%	Gleichung = \dots
%\end{equation}

% SI-Einheiten (Beispiele)
%\SI{1000}{\ohm}
%\SI{10}{\micro\metre}
%\si{\micro\metre}
%\si\micro

% Zeitliche Ableitungen
%$\dot{x}$
%$\ddot{x}$

\begin{document}
\maketitle
\vspace{.5cm}
\setcounter{page}{0}
\thispagestyle{empty}
\newpage

% Grundlagen
\tableofcontents
\section{Allgemeines}
Festkörper lassen sich in verschiedene Kategorien einteilen, amorphe Festkörper, ideale Kristalle (Einkristalle).\\
Fast alle Festkörpereigenschaften lassen sich aus der Schrödinger-Gleichung herleiten.\\
Typische Festkörper besitzen eine Größe von mehreren $ \SI{}{\centi\meter^3} \hat{=} 10^{23}$ Atomen oder mehr.\\
Ebenfalls wichtig: Pauli-Prinzip, Maxwellsche Gleichungen, Konzepte der Thermodynamik, Statistische Mechanik.\\
In \emph{Fluiden} ändert sich die Position der Atome zeitlich, in Festkörpern bleibt sie weitgehend gleich, jeweils in einem \emph{lokalen Minimum} der potentiellen Energie. Ein Auslenken eines Atoms führt aufgrund von Kopplung auch zu Schwingungen der anderen Atome. 
\section{Bindung im Festkörper}
\begin{figure}[tb]
\centering
\includegraphics[width=0.7\linewidth]{Lennard_Jones}
\caption{Lennard-Jones-Potential. Beschreibt das Potential zwischen zwei neutralen Atomen mit abgeschlossenen Elektronenschalen.}
\label{fig:Lennard_Jones}
\end{figure}
\subsection{Bindungstypen}
Es gibt 5 Grundtypen von Bindungen, sie unterscheiden sich hauptsächlich in der räumlichen Verteilung der beteiligten Elektronen. Meist trifft man jedoch Mischformen an.\\
Elektrostatische Anziehung zw. Elektronen (negativ) \& Atomkernen (positiv) einzige Ursache für Zusammenhalt von FK.
\paragraph*{Haupttypen kristalliner Bindungskräfte:}
\begin{itemize}
	\item van der Waals-Kräfte: schwach, neutrale Atome, durch Fluktuationen in den Ladungsverteilungen der Atome
	\item ionisch: anziehende elektrostatische Kräfte zw. positiv \& negativ geladenen Ionen
	\item metallisch: Alkaliatome haben Valenzelektron an gemeinsamen Elektronensee abgegeben
	\item kovalent: neutrale Atome durch Überlapp ihrer Elektronenverteilung zusammen gehalten
	\end{itemize}
\paragraph*{Bindungsenergie $\mathrm{oder}$ Gitterenergie:} \begin{itemize}
	\item Arbeit, die zur Zerlegung des Festkörpers (im Grundzustand) in seine Bestandteile (d.h. Ionen, Atome) aufgewendet werden muss. 
	\item Maß für die Stärke der Wechselwirkung (elektrostatische Kräfte) und Zahl der wechselwirkenden Atome.
\end{itemize}
\paragraph*{Lennard-Jones-Potential:} \begin{itemize}
	\item Ergibt sich aus Addition von van-der-Waals-Potential (anziehend, 1) und abstoßenden Potential von zwei neutralen Atomen/Molekülen mit abgeschlossenen Elektronenschalen (2): $ \varphi(r) = \underbrace{\frac{A}{r^{12}}}_{(1)} \underbrace{- \frac{B}{r^{6}}}_{(2)} = 4\varepsilon \left[\left(\frac{\sigma}{r}\right)^{12} - \left(\frac{\sigma}{r}\right)^6\right]$
	\item Dargestellt in \autoref{fig:Lennard_Jones}, bei starker Annäherung geht Anziehung in Abstoßung über und verhindert ein Durchdringen der Atome. Grund: Pauli-Prinzip $ \Rightarrow $ Atome verhalten sich fast wie harte Kugeln.
\end{itemize}
\subsubsection{van-der-Waals-Bindung (v-d-W-B.)}
Edelgasatome mit ihren abgeschlossenen Elektronenschalen gehen i. d. R. keine Bindung mit Nachbaratomen ein \begin{itemize}[label=]
	\item \begin{itemize}[label=$\Rightarrow$]
		\item auch in FK Kugelgestalt.
		\item bilden Kristalle hoher Symmetrie.
	\end{itemize}
\end{itemize}

Wirken zwischen sämtlichen Atomen, beruhen auf elektrischer \emph{Dipol-Dipol-Wechselwirkung}. Auch wenn Edelgasgasatome (= kugelsymmetrische Atome) kein permanentes Dipolmoment besitzen, tritt die v-d-W-B. auch hier auf. Dies liegt daran, dass die Ladungsverteilung der Atome fluktuiert.
\subsubsection{Ionenbindung}
Auch \emph{heteropolare Bindung} genannt, folgt aus Elektronentransfers zwischen Atomen, die zur Ionenbildung führen.
\paragraph*{Madelung-Konstante $ \alpha $:}
\subsubsection{Kovalente Bindung}
\begin{itemize}
	\item klassische Elektronenpaar- oder gleichgeladene Bindung der Chemie
	\item sehr stark, vergleichbar mit Ionenbindung, stark gerichtet
	\item Normalerweise von 2 Elektronen gebildet, je 1 Elektron pro Atom. Elektronen befinden sich zwischen Atomen mit antiparallelen Spins
	\item Bindung hängt von relativer Spinstellung ab, da Ladungsverteilung je nach Spinorientierung zueinander ändert (Pauli-Prinzip!)\\
	$ \rightarrow $ Spinabhängige Coulomb-Energie heißt \textbf{Austauschwechselwirkung}.
	\item Pauli-Prinzip $ \rightarrow $ abstoßend bei vollständig gefüllten Schalen, teilweise gefüllt: Orbitale überlappen und verkürzen Bindungslänge, ohne dass Anregung in höhere Energieniveaus notwendig 
	\item Übergang ionisch $ \leftrightarrow $ kovalent: kontinuierlich
\end{itemize}
\subsubsection{Metallische Bindung}
\paragraph*{Wigner-Seitz-Radius:}
\subsubsection{Wasserstoffbrückenbindung}
Tritt in wasserstoffhaltigen Substanzen auf.
\section{Struktur der Festkörper}
\subsection{Herstellung}
Ohne besondere Vorkehrungen: Beim Abkühlen einer Schmelze i. A. $ \rightarrow $ polykristalline Festkörper. Einkristalle sind anisotrop (d.h. ihre Eigenschaften sind richtungsabhängig), während \enquote{Polykristalle} isotrope phys. Eigenschaften besitzen.\\
Einkristalle sind wichtig für Materialwissenschaften, Halbleitertechnik (Silizium) und Nachrichtentechnik (piezoelektrische Einkristalle).
\subsection{Ordnung und Struktur der Kristalle}
Idealkristalle und amorphe Festkörper beschreiben zwei extreme Fälle, zwischen denen ein breites Spektrum an Festkörpern mit teilweiser Ordnung existiert.
\paragraph*{Nahordnung:}
\paragraph*{Fernordnung:} Ordnung in Festkörpern über große Abstände hinweg. Ist gegeben bei idealen Einkristallen.
\paragraph*{Wertigkeit:}
\paragraph*{Basis:} \begin{itemize}
	\item  Bezeichnet identische, gleich orientierte Atomgruppe, aus denen sich ein idealer Festkörper zusammensetzt.
	\item wird dreidimensional periodisch aneinander gereiht.
	\item Zahl der Atome in dieser Gruppe kann von 1 (viele Metalle) bis $ >10^4 $ (Proteinkristalle) bestehen.
	\item Wird jeder Basis ein Punkt im Raum zugeordnet, so entsteht ein \emph{Punktgitter}, welches eine mathematische Beschreibung ermöglicht. 
	\item Wahl der Basis ist \textbf{nicht} eindeutig.
\end{itemize}
\paragraph*{Symmetrie von Kristallen:} Lässt sich unterscheiden in \emph{Translationssymmetrie} (Struktur als Ganzes verschoben) und Punktsymmetrie (mind. 1 Punkt in Raum festhalten). Die Translationssymmetrie wird am häufigsten verwendet.
\paragraph*{Translations- $\mathrm{oder}$ Gittervektor $ \textbf{R} $:} Im Idealkristall wiederholt sich die Umgebung $ \mathcal U $ für $ \mathcal U(r) = \mathcal{U} (\textbf{r}+\textbf{R})$ mit dem beliebigen Ortsvektor $ \textbf{r} $. Die Translation zu einem äquivalenten Punkt ist $ \textbf{R} $, dabei gilt \begin{align*}
\textbf{R} = n_1 \textbf{a} + n_2 \textbf{b} + n_3 \textbf{c}\mathrm{.}
\end{align*}
$ \textbf{a, b, c} $ werden \emph{fundamentale Translationsvektoren} oder \emph{Basisvektoren} genannt und spannen ein schiefwinkliges KS auf, welches an die Gittersymmetrie angepasst ist. Die Längen der Basisvektoren werden als \emph{Gitterkonstanten} bezeichnet. $ n_{1,2,3} $ sind ganzzahlig. Die Wahl der Basisvektoren ist nicht eindeutig!
\paragraph*{Elementarzelle $\mathrm{oder}$ Einheitszelle:} \begin{itemize}
	\item Beschreibt Parallelepiped, welches von Basisvektoren aufgespannt wird.
	\item Der Raum kann lückenlos gefüllt werden, wenn sie aneinander gereiht werden.
\end{itemize}
\paragraph*{Zähligkeit $ n $:}\begin{itemize}
	\item gibt an, wie oft bei einer Drehung des Kristalls um $ 2\pi $ Deckungsgleichheit auftritt
	\item $ n\in \left\{1,2,3,4,6\right\} $, Rest lässt sich nicht raumfüllend anordnen
\end{itemize}
\paragraph*{Bravais-Gitter:}
\begin{itemize}
	\item primitive EZ stellen oft nicht komplette Gittersymmetrie dar $ \rightarrow $ stattdessen Bravais-Gitter
	\item $ 14  $ 3D-Bravais-Gitter, 7 davon mit nicht-primitiver EZ\\
	\includegraphics[width=\linewidth]{bravais}
\end{itemize}
\paragraph*{Packungsverhältnis:}
\begin{itemize}
	\item \% des Raums, der von identischen, sich berührenden Kugeln an den Gitterpunkten ausgefüllt ist. Kantenlänge des Würfels ist Gitterkonstante $ a $.
\end{itemize}
\paragraph*{Kubisches Gittersystem:}
\begin{itemize}
	\item besteht aus 3 Gittern\\
	\hspace*{-2cm}\begin{tabular}{cccc}
		&\includegraphics[width=0.2\linewidth]{sc} & \includegraphics[width=0.2\linewidth]{bcc} & \includegraphics[width=0.2\linewidth]{fcc} \\
		& einfach kubisch & kubisch raumzentriert & kubisch flächenzentriert\\
		&\textbf{sc} & \textbf{bcc} & \textbf{fcc}\\
		& simple cubic & body-centered cubic & face-centered cubic\\
		V(prim. Zelle) & $ a^3 $ & $ \frac{a^3}{2} $ & $ \frac{a^3}{4} $ \\
		Abst. n. Nachb. & $ a $ & $ \frac{\sqrt{3}a}{2} $ & $ \frac{a}{\sqrt{2}} $\\
		Packungsv. & $ \SI{0,52}{} $ & $ \SI{0,68}{} $ & $ \SI{0,74}{} $
	\end{tabular}
	\item Sonderfall Diamant: gleichartige Atome, aber prim. Elementarzelle 2 Atome
\end{itemize}
\paragraph*{Hexagonal dichteste Kugelpackung (hcp):}
\begin{itemize}
	\item Packungsverhältnis $ \SI{0,74}{} $
	\item 12 nächste Nachbarn (Maximalwert)
	\item Beispiel: viele Metalle
\end{itemize}
\paragraph*{Wigner-Seitz-Zelle:}
\begin{itemize}
	\item primitive Zelle, Zentrum ist ein Gitterpunkt
	\item enthält Symmetrie und ermöglicht lückenloses Füllen
	\item umschließt Raum, der Aufgitterpunkt näher als jedem anderen Gitterpunkt ist
	\item \textbf{Konstruktion:} 2D: kleinste Fläche, die bei Zeichnen von Mittelsenkrechten der Verbindung zu den Nachbaratomen eingeschlossen wird
	\begin{center}
		\includegraphics[width=0.3\linewidth]{wignerseitz}
	\end{center}
\end{itemize}
\section{Strukturbestimmung}
HIER: Beugungsexperiment = Streuexperiment
\begin{itemize}
	\item Untersuchung von \textbf{Oberflächen:} Mikroskopie (Elektronen-, Rastertunnel-, Rasterkraft-) $ \Rightarrow $ wichtig für Oberflächenphysik
	\item \textbf{Innerer Aufbau} von FK: Streuexperimente mit kohärenter Streuung \begin{itemize}
		\item \textbf{massive FK:} Röntgen- \& Neutronenstrahlen
		\item \textbf{Oberflächen, dünne Filme:} Elektronen- und Atomstrahlen
		\item Neutronen ww. i. d. R. mit den Kernen, der Rest mit Elektronenhüllen 
	\end{itemize}
	\item Aus Optik: Beugung stark für $ \lambda\approx $ Abmessungen beugendes Objekt\\
	$ \Rightarrow $ unterschiedliche Energien je nach Strahlung
	\item \textbf{Streuprozesse} werden unterschieden in
	\begin{itemize}
		\item \textbf{elastisch:} Energie der Strahlung konst., da elektronische \& vibronische Zustände vom FK unbeeinflusst $ \Rightarrow $ zur Strukturbestimmung
		\item \textbf{inelastisch:}  Energieaustausch gestreute Teilchen $ \leftrightarrow $ FK $ \Rightarrow $ Untersuchung von Anregungszuständen
	\end{itemize} 
\end{itemize}
\paragraph*{Reziprokes Gitter:}\begin{itemize}
	\item Neues Koordinatensystem, aufgespannt von Basisvektoren $ \vec b_{1,2,3} $
	\item Kristalle translationssymmetrisch $ \rightarrow $ Stromdichteverteilung $ \rho(\vec r) $ periodisch\\
	$ \Rightarrow $ Symmetrie nutzen $ \rightarrow \rho(\vec r)$ in Fourier-Reihe entwickeln $ \rightarrow $ einzelne Fourierkomponenten betrachten
	\item F-R von $ \rho(\vec r): \rho(\vec r) = \sum\limits_{h,k,l} \rho_{hkl}e^{i \vec G_{hkl} \cdot \vec r}~~, h, k, l \in \mathbb Z ~, \rho_{hkl} = \frac{1}{V_Z} \int_{V_Z} \rho(\vec r) e^{-i\vec G_{hkl} \cdot \vec r}\mathrm{d}V $ Integration über Volumen der primitiven Elementarzelle $ V_Z $
	\item $ \vec G_{hkl} = h \vec b_1 + k \vec b_2 + l \vec b_3 $ ist Punkt von reziprokem Gitter
	\item Es gilt $ \vec b_i \cdot  \vec a_j = 2 \pi \delta_{ij} \Rightarrow \vec b_1 = \frac{2\pi}{V_Z} \left(\vec a_2 \times \vec a_3\right), \vec b_2 = \frac{2\pi}{V_Z} \left(\vec a_3 \times \vec a_1\right) , \vec b_3 = \frac{2\pi}{V_Z} \left(\vec a_1 \times \vec a_2\right)  $
	\item Volumen der primitiven Elementarzelle: Reales Gitter: $ V_Z = \left(\vec a_1 \times \vec a_2\right)\cdot \vec a_3 $, Reziprokes Gitter: $ \left(\vec b_1 \times \vec b_2\right)\cdot \vec b_3 = \frac{(2\pi)^3}{V_Z} $
	\item Vektoren haben \emph{inverse Länge} $ \rightarrow $ existieren im \emph{$ \vec k $-Raum}, wegen $ \vec p = \hbar \vec k  $ auch \emph{Impulsraum} genannt.
\end{itemize}
\paragraph*{Brillouin-Zone:}\begin{itemize}
	\item Wigner-Seitz-Zelle des reziproken Gitters heißt 1. Brillouin-Zone.
	\item Geometrische Veranschaulichung der Beugungsbedingung $ 2 \vec k \cdot \vec G  = G^2$
	\item Wichtig für Gitterdynamik, Beschreibung von Elektronenbewegung
	\item Konstruktion analog zur Wigner-Seitz-Zelle
	\item Höhere Ordnung: auch weiter entfernte Punkte miteinbeziehen.
	\item Mittelpunkt der 1. Zone heißt $ \Gamma $-Punkt.
	\item Polyeder füllen den reziproken Raum lückenlos.
\end{itemize}
\paragraph*{Kristall-, Gitter - $\mathrm{oder}$ Netzebene:} \begin{itemize}
	\item Ebene, die durch Gitterpunkte aufgespannt wird.
	\item Translationsinvarianz $ \Rightarrow $ (unendlich) viele gleiche Ebenen parallel
\end{itemize}
\paragraph*{Millersche Indizes:}
\begin{itemize}
	\item Reziproke Längen zur Charakterisierung von Schar von parallelen Ebenen
	\item \textbf{Vorgehen:} Achsenabschnitte suchen, Kehrwert bilden, mit gleichem Faktor multiplizieren, sodass möglichst kleine ganze Zahlen entstehen.
	\item \textbf{Bsp.:} Achsenabschnitte bei $ \frac14 $ ($\vec a_1$), $ \frac12 $ ($\vec a_2$), $\infty $ ($\vec a_3$) $ \Rightarrow (hkl) = (120) $
	\item Negative Achsenabschnitte: $ -1 \hat = \bar 1 $, z. B.: $ (\bar 120) $
	\item Auch Richtungsangaben möglich: $ [hkl]= h \vec a_1 + k \vec a_2 + l \vec a_3 $, $ \perp $ auf $ (hkl) $
	\item Bezeichnen der gleichwertigen Ebenen: $ \left<hkl\right> $ oder $ \left\{hkl\right\} $
	\item Relevant: Meist Ebenen mit geringen Indizes, zeigen grundlegende Periodizitäten.
	\item Sonderfall hexagonale Kristalle: unerwartete Gleichwertigkeiten (z. B. $ (100), (1\bar 1 0) $) $ \rightarrow $ zusätzlicher Index $ i= -(h+k) \Rightarrow (hkil)$
	\item Jedem $ (hkl) $ ist ein reziproker Gittervektor $ G_{hkl} $ zugeordnet.
\end{itemize}
\paragraph*{Streuung an Kristallen:}
\begin{itemize}
	\item Als Näherung: Eintreffende und gestreute Welle als ebene Welle
	\item Zusammenhang zwischen $ \rho(\vec r) $ und Streuintensität herstellen
	\item Relevant: Phasendifferenz, bei $ k=k_0 $ (elastische Streuung): $ \Delta \varphi = \left(\vec k - \vec k_0\right)\cdot \vec r $, $ \vec k_0 $ eintreffender Wellenvektor, $ \vec k $ ausgehender
	\item \textbf{Streubedingung:} $ \vec K = \vec k - \vec k_0 = \vec G $, $ \vec K $ Streuvektor \begin{itemize}
		\item Alle anderen Fälle: durch Interferenz Auslöschung
		\item Bei endlichem Probenvolumen: $  \vec K \approx \vec G $, Streubedingung etwas aufgeweicht
	\end{itemize}
\end{itemize}
\paragraph*{Ewald-Kugel:}
\begin{itemize}
	\item Geometrische Konstruktion im Impulsraum um passende Orientierung des Kristalls für Reflexe zu ermitteln.\\
	\includegraphics[width=\linewidth]{Ewaldkugel.png}
	\item \textbf{Konstruktion in 2D:} \begin{enumerate}
		\item Reziprokes Gitter aufzeichnen,
		\item $ \vec k_0 $ so einzeichnen, dass der Endpunkt des Vektors mit dem Ursprung des reziproken Gitters übereinstimmt. 
		\item Kreis (in 3D: Kugel) um Anfangspunkt des Vektors mit Radius $ k_0 $ ziehen.
		\item Streubedingung erfüllt, wenn ein Punkt des reziproken Gitters die Ewald-Kugel berührt. Dorthin zeigt $ \vec k $ aus dem Kreismittelpunkt.
		\item  Gebeugter Strahl tritt in Richtung $ \vec k $ auf.
	\end{enumerate}
	\begin{tabular}{|c|c|}
		  &  \\ 
		endliche Abmessungen  & Punkte verschmiert, vergrößert darstellen \\ 
		begrenzte Strahl-Eindringtiefe  & s. o., oder: Schichtdicke der Ewald-Kugel erhöhen \\ 
		  &  \\ 
		  &  \\ 
		\hline 
	\end{tabular}  
\end{itemize}
\paragraph*{Bragg-Bedingung:}
\paragraph*{Strukturfaktor:}
\paragraph*{Debye-Waller-Faktor:}

\subsection{Streuung an Kristallen}
\subsection{Experimentelle Methoden}
\includegraphics[width=\linewidth]{monochromatisierung}
\paragraph*{Drehkristallverfahren:} \begin{itemize}
		\item Strahlung monochromatisch, Einkristalle, Einfallswinkel durch Kristalldrehung variiert
		\item Zur Bestimmung von Gitterkonstanten
		\item Bei Drehung um Drehachse dreht sich auch reziprokes Gitter
		\item Reflexion, wenn Bragg-Bedingung erfüllt (Gitterpunkt auf Ewaldkugel-Oberfläche)  \begin{center}
		\includegraphics[width=\linewidth]{Drehkristallmethode}
	\end{center}
	\end{itemize}
\paragraph*{Pulvermethode $ \mathrm{oder} $ Debye-Scherrer-Verfahren:}
\includegraphics[width=\linewidth]{Debye-Scherrer}
\paragraph*{Laue-Verfahren:}
\begin{itemize}
	\item \begin{center}
		\includegraphics[width=\linewidth]{Laue-Aufnahme}
	\end{center} 
\end{itemize}

\section{Strukturelle Defekte}
\textcolor{red}{\bf{\fbox{ACHTUNG:}} Kapitel übersprungen}\\
Bisher: von idealen Kristallen/amorphen Festkörpern ausgegangen. Tatsächlich sind viele physikalische Eigenschaften von realen Materialien durch Defekte beeinflusst.
\subsection{Punktdefekte}
\subsection{Ausgedehnte Effekte}
\section{Gitterdynamik}
\begin{itemize}
	\item Festkörpereigenschaften meist zurückzuführen auf:
	\begin{itemize}
		\item \textbf{Gitterdynamische} Eigenschaften:  Atombewegung um Gleichgewichtslage
		\item \textbf{Elektronische} Eigenschaften: Bewegung fast freier Elektronen
		\item Voneinander weitgehend unabhängig, da Elektronen deutlich schneller als schwere Atomkerne $ \Rightarrow $ \enquote{instantan} neue Elektronenverteilung, falls Atome aus Gleichgewichtslage verschoben (Erhöhung Gesamtenergie). Bei Rückkehr zu Ausgangslage wird aufgewandte Energie zurückgewonnen, Elektronen bleiben dauerhaft im Grundzustand.\\
		$ \Rightarrow $ Untersysteme können getrennt betrachtet werden, heißt \textbf{adiabatische} oder \textbf{Born-Oppenheimer-Näherung}.
	\end{itemize}
\end{itemize}
\subsection{Elastische Eigenschaften}
\begin{itemize}
	\item Makroskopische Betrachtung (nicht-atomare Ebene), \textbf{elastisches Kontinuum}
	\item Annahme: Verformungen klein, sodass linearer Zusammenhang Kraft $ \leftrightarrow $ Verformung
\end{itemize}
\paragraph*{$\mathrm{(mechanische)}$ Spannung $ [\sigma]$:} \begin{itemize}
	\item $[\bf{\sigma}]$ Tensor,  $ \sigma_{ij} $, Index $ i $: Kraftrichtung, Index $ j $: Angriffsfläche der Kraft
	\item Komponenten bestimmen: Probe in kleinen Würfel zerschneiden (Kanten $ \parallel $ Achsen kartesisches Koordinatensystem)\\
	\begin{center}
		\includegraphics[width=0.4\linewidth]{Spannungstensor}
	\end{center}
	\item Vorzeichen: Druck $ \hat = -$, Zug $ \hat = +$ 
	\item Tensor \textbf{symmetrisch}, falls kein Drehmoment (im Gleichgewicht muss eine Gegenkraft herrschen) $ \rightarrow $ 6 statt 9 unabhängige Komponenten
\end{itemize}
\paragraph*{Dehnungstensor $ [\textbf{e}] $:}
\begin{itemize}
	\item beschreibt Verformung
	\item \textbf{Hookesches Gesetz} verknüpft [$ \textbf{e} $] und [$ \sigma $]: $ \sigma_{ij}= \sum\limits_{kl} c_{ijkl}e_{kl} $, $ c_{ijkl} $ 4-stufiger \textbf{Elastizitätstensor}, aus Symmetriegründen 21 statt 81 unabh. Komponenten.
\end{itemize}
\paragraph*{Schallwellen:} \begin{itemize}
	\item breiten sich auch in Festkörpern aus.
	\item Wellen lassen sich anhand des Zusammenhangs zwischen $ \vec u $ (Auslenkung von FK-Stück aus Ruhelage) und $ \vec q $ (Wellenvektor der Schallwelle):
	\begin{itemize}
		\item \textbf{longitudinal:} $ \vec u \parallel\vec q $\\
		$ \rightarrow $ \emph{quasi-longitudinal}: longitudinal überwiegt
		\item \textbf{transversal:} $ \vec u \perp \vec q $\\
		$ \rightarrow $ \emph{quasi-transversal}: transversal überwiegt
		\item \textbf{Dispersionsrelation} linear $ \rightarrow $ Zusammenhang zwischen Frequenz und Wellenzahl linear, Proportionalitätsfaktor: FK-spezifische Schallgeschwindigkeit\\
		$ \Rightarrow $ Schallausbreitung in elast. Kontinua dispersionsfrei
		\item Erzeugung von ebenen Schallwellen:
			\begin{center}
				\includegraphics[width=0.4\textwidth]{Ultraschallechos}
			\end{center}
			Beobachtet: \emph{Ultraschalldämpfung}, exponentieller Abfall\\
			eignet sich zur Bestimmung des Elastizitätstensors
	\end{itemize}
\end{itemize}
\subsection{Gitterschwingungen}
\begin{itemize}
	\item ein- und zweiatomige Basis unterscheiden sich qualitativ. Größere Basis rechnerisch aufwändig, vom Konzept einfach.
	\item \textbf{Vereinfachung: lineare Kette}, besonders realitätsnah  bei Ausbreitung in Richtung hoher Symmetrie: \begin{itemize}
		\item longitudinal: alle Kräfte nicht in Ausbreitungsrichtung mitteln sich aus Symmetriegründen weg, resultierende Kraft nur in Ausbreitungsrichtung.
		\item transversal: nur Kräfte $ \perp $ Ausbreitungsrichtung, in longitudinaler Richtung kompensieren sich Kräfte
		\item 1D-Kette reicht, da in einer Netzebene alle Atome gleiche Bewegung
		\item In den meisten Festkörpern fällt WW so schnell ab, dass nur nächste oder übernächste Nachbarn beachtet werden müssen.
		\item Im Fall kleiner Auslenkungen kann die rückstellende Kraft bei Auslenkung eines Atoms mit dem Hookesschen Gesetz (Feder) beschrieben werden.  \enquote{Federkonstante} $ D $ kann für transversal und longitudinal verschieden sein!
	\end{itemize}
\end{itemize}
\paragraph*{Einatomige Basis:}
\begin{itemize}
	\item wird nur die 1D-Kette mit nächsten Nachbarn berücksichtigt, so ergibt sich für ein Atom die Bewegungsgleichung \begin{align*}
	M \ddot{u}_n (t) = D \left(u_{n+1} + u_{n-1} - 2 u_n\right)\\
	\text{mit Ansatz } u_{n} = U e^{i\left(\omega t - kna\right)} \Rightarrow \omega = 2 \sqrt{\frac{D}{M}} \left|\sin\left(\frac{ka}{2}\right)\right|
	\end{align*}
	$ a $ Abstand Netzebenen, n aktuelles Atom, $ u $ Verschiebung, $ M $ Masse des Referenzatoms, $ k $ Wellenvektor.
	\begin{center}
		\includegraphics[width=0.4\linewidth]{einatomige_Basis}
	\end{center}
	\item \textbf{1. Brillouin-Zone enthält sämtliche Informationen} ($ - \frac{\pi}{a} < k \le \frac{\pi}{a}$) (Grund: aus Ansatz, benachbarte Atome max. Phasenunterschied $ 2\pi $)
	\item Grenzfallbetrachtung:
	\begin{itemize}
		\item \textbf{langwellig,} $ q\rightarrow 0 $: linearer Zusammenhang zwischen $ \omega $ und $ k $.
		\item \textbf{kurzwellig,} $ q\rightarrow \pm \frac{\pi}{a} $: NachbarAtome schwingen gegenphasig, stehende Welle.
	\end{itemize}
\end{itemize}
\paragraph*{Zweiatomige Basis:}
\begin{itemize}
	\item Bewegungsgleichung bei WW nur mit nächsten Nachbarn: \begin{align*}
	M_1 \ddot{u}_n (t) = D (v_n + v_{n-1} - 2 u_n) ~\mathrm{ und }~ M_2 \ddot{v}_n (t) = D \left(u_n + u_{n-1} - 2 v_n\right)\\
	\text{mit Ansätzen } u_n(t) = u_0 \cdot e^{i(nka-\omega t)} \text{ und } v_n(t) = v_0 \cdot e^{i(nka-\omega t)}\\
	 \Rightarrow \omega^2_{\pm} = D \underbrace{\left(\frac{1}{M_1} + \frac{1}{M_2}\right)}_{\text{reduzierte Masse }\mu} \pm \sqrt{\left(\frac{1}{M_1} + \frac{1}{M_2}\right)^2 - \frac{4}{M_1M_2}\cdot \sin^2 \left(\frac{ka}{2}\right)} 
	\end{align*}
	$ \omega_+ $ heißt \textbf{optischer Zweig}, $ \omega_- $ \textbf{akustischer Zweig}.
	\begin{center}
		\includegraphics[width=0.5\linewidth]{zweiatomige_Basis}
	\end{center}
	\item \textbf{langwellig} $ q\rightarrow 0 $:\begin{itemize}
		\item optischer Zweig kaum von Wellenvektor abhängig. Stark unterschiedliche Frequenzen, Grund: akustisch: benachbarte Atome fast die gleiche Phase, optisch: die Atome beider Untergitter schwingen gegenläufig.
		\item Falls beide Atomsorten entgegengesetzte Ladungen besitzen, entsteht oszillierendes Dipolmoment, können an em. Wellen ankoppeln $ \rightarrow $ \textbf{infrarot-aktiv} (z. B. Ionenkristalle).
	\end{itemize} 
	\item \textbf{langwellig} $ q \rightarrow \frac{\pi}{a} $:  je nach betrachtetem Zweig ist entweder das leichte oder das schwere Untergitter in Ruhe, das andere schwingt.
	\item \textbf{Frequenzlücke:} Zwischen beiden Zweigen \textbf{verbotene Zone} $ \rightarrow $ dort keine Eigenschwingungen $ \rightarrow $ Wellen klingen exponentiell ab.
	\item \textbf{Immer:} 3 akustische Zweige, 1 longitudinal, 2 transversal (Polarisation $ \perp $)
	\item \textbf{Allgemein:} Basis mit $ p $ Atomen hat $ 3p $ Zweige, 3 akustische, $ 3p-3 $ optische. Optisch: Doppelt so viele transversal wie longitudinal.
	\end{itemize}
\paragraph*{Inelastische Streuung}\begin{itemize}
	\item \textbf{Streubedingung} $ \vec K \mp \vec q= \vec G $. Es ergeben sich \textbf{Quasiimpuls-} und \textbf{Energieerhaltung:} \begin{align*}
	\hbar \omega = \hbar \omega_0 \pm \hbar \omega_{\vec{q}}\\
	\hbar \vec k = \hbar \vec k_0 \pm \hbar \vec q + \hbar \vec G
	\end{align*}
	\item Einfallendes Röntgenquant/Neutron/Elektron ww mit Gitter und erzeugt/vernichtet 1 Schwingungsquant (\textbf{Phonon}) mit dem Impuls $ \hbar \vec q $  (\textbf{Quasiimpuls/Kristallimpuls})und der Energie $ \hbar \omega_{\vec{q}} $
	\item \textbf{Phononerzeugung:}
	\begin{center}
		\includegraphics[width=0.4\linewidth]{ewald_phononerzeugung}
	\end{center}
	Bei Phononvernichtung wäre der gestreute Wellenvektor $ \vec k $ außerhalb der Ewaldkugel.
	\item \textbf{$ \rightarrow $ Gitterwellen besitzen auch Teilchencharakter, quantisiert!} $ E_q = \left(n_q+ \frac12\right) \hbar \omega_q $, $ n_q =$ Anzahl der Phononen.
	\item Bezeichnung Phononenzweige TO, LO, TA, LA transversal/longitudinal optischer/akustischer Zweig
\end{itemize}
\paragraph*{Phononen:} \begin{itemize}
	\item Phonon  nicht einzelnem Atom zugeordnet, \textbf{jedes} Atom trägt zu jedem Phonon bei.
	\item kein echter Impuls $ \rightarrow $ keine fundamentalen Teilchen $ \rightarrow $ \emph{Quasiteilchen}. ACHTUNG: Impuls gestreuter Teilchen verändert sich \emph{wirklich}, gesamter Festkörper übernimmt Impulsübertrag.
	\item Phononendispersionsbestimmung: \textbf{kohärente inelatische Streuung}, Neutronen und Röntgen.
	\end{itemize}
\paragraph*{Rayleigh-Streuung:} \begin{itemize}
	\item WW von Licht mit FK ohne Phononenbeteiligung
	\item elastischer Streuprozess, \emph{keine} Frequenzverschiebung
	\item Aus Streubedingung $ \Rightarrow $ $ \vec k = \vec k_0 \Rightarrow $ nur Vorwärtsstreuung!
	\item Tritt in realen Kristallen auf, Defekte als Streuzentren
\end{itemize}
\paragraph*{(Anti-)Stokes-Prozess:}\begin{itemize}
	\item WW Licht mit FK \textbf{MIT} Phononenbeteiligung\\
	\begin{center}
		\includegraphics[width=0.8\linewidth]{stokes_prozess}
	\end{center}
	\item \textbf{Stokes:} Phononerzeugung $ \Rightarrow $ gestreutes Licht größere Wellenlänge
	\item \textbf{Anti-Stokes:} Phononvernichtung $ \Rightarrow $ gestreutes Licht kleineres $ \lambda $
	\item kleiner Wellenvektor $ \Rightarrow $ nur Phononen in unmittelb. Umgebung beteiligt
	\item Wellenzahl Phononen: $ q = 2k_0 \sin \left(\frac{\theta}{2}\right) $
	\end{itemize}
\paragraph*{Raman-Streuung:}\begin{itemize}
	\item Lichtstreuung an \textbf{optischen} Phononen
	\item Lichtverschiebung kaum  beobachtungsrichtungsabh., da Phononenfrequenz kaum $ \lambda $-Abhängigkeit
	\item gestreutes Laserlicht wird von Doppelmonochromator analysiert
	\item \textbf{Typische Verschiebung}: ca. $ \SI{\pm10}{\tera\hertz} $
	\begin{center}
		\includegraphics[width=0.4\linewidth]{raman_spektren}
	\end{center}
	\item Lage und Breite \textbf{temperaturabhängig}, Grund: leichte Anharmonizität des Gitters
	\item zusätzlich zu beobachten: \textbf{Infrarotabsorption}, schließen sich oft gegenseitig aus
\end{itemize}
\paragraph*{Brillouin-Streuung:}\begin{itemize}
	\item $ \nu_\text{gestreut} $ vom Streuwinkel $ \vartheta $ abhängig
	\item Streulicht von frequenzstabilisiertem Laser wird mit \emph{Fabry-Perot-Interferometer} analysiert
	\item \textbf{Typische Verschiebung:} ca. $ \SI{\pm20}{\giga\hertz} $
		\begin{center}
			\includegraphics[width=0.4\linewidth]{brillouin_spektrum}
		\end{center}
\end{itemize}
\subsection{Spezifische Wärmekapazität}
\begin{itemize}
	\item Spezifische Wärme bei konst. Druck $ C_p $: wird experimentell bestimmt
	\item Spezifische Wärme bei konst. Volumen $ C_V $: $ \left(\frac{\partial U}{\partial T}\right)_V $, zur theoretischen Beschreibung
	\item Zusammenhang: $ C_p - C_V = 9\alpha^2 VT/\kappa $, $ \alpha $ lin. therm.  Ausdehnungskoeff., $ \kappa $ Kompressibilität
	\item $ C_p $ und $ C_V $ in FK kaum unterschiedlich
\end{itemize}
\paragraph*{Dulong-Petit-Gesetz}
\begin{itemize}
	\item beschreibt $ C_V $ bei hohen Temperaturen in FK
	\item $ C_V = 3 N_A k_B = 3 R_m $, $ R_m $ Gaskonst., $ N_A $ Avogadro-Konst.
\end{itemize}
\textcolor{red}{\bf{\fbox{ACHTUNG:}} Zustandsdichten-Abschnitte übersprungen, nachfolgendes zu ungenau?}
\paragraph*{Einstein-Modell:}
\begin{itemize}
	\item nur optische Phononen berücksichtigt
	\item dispersionslos $ \Rightarrow \omega(\vec k) =  $konst.
	\item polarisationsunabh.
	\item $ C_V= \left(\frac{\partial U}{\partial T}\right)_V = 3N\left(\frac{\Theta_E}{T}\right)^2k_B \cdot \frac{e^{\frac{\Theta_E}{T}}}{\left(e^{\frac{\Theta_E}{T}}-1\right)^2} $, $ \Theta_E = \frac{\hbar \omega_o}{k_B}$ Einstein-Temperatur\\
	$ \Rightarrow T\rightarrow\infty$: Dulong-Petit, $ C_V(T\rightarrow 0) = 0 $, Anstieg exponentiell 
\end{itemize}
\paragraph*{Debye-Modell:}\begin{itemize}
	\item nur akustische Phononen berücksichtigt
	\item Näherung: Dispersion $ \omega(\vec k) = \omega(k)=vk $
	\item \textbf{Debyesches $ T^3 $-Gesetz:} $ C_V = \left(\frac{\partial U}{\partial T}\right)_V = \frac{12 \pi^4}{5} N k_B \left(\frac{T}{\Theta_D}\right)^3$,\\
	$ \Theta_D = \hbar \omega_D/k_B$ \textbf{Debye-Temperatur}, $ N $ Zahl der Zustände\\
	$ \Rightarrow T\rightarrow\infty$: Dulong-Petit, $ C_V(T\rightarrow 0) = 0 $, Anstieg $ T^3 $
	\item Beispielwerte $ \Theta_D $: C $ \SI{2230}{\kelvin} $, Au $ \SI{164}{\kelvin} $
	\item Übereinstimmung: sehr gut für kleine und große Temperaturen, ungenau im Zwischenbereich
	\item \textbf{2D-Systeme}: $ C_V \propto T^2 $ 
	\end{itemize}
\section{Anharmonische Gittereigenschaften}
\begin{itemize}
	\item Wäre Potential komplett harmonisch, so würde sich kein thermisches Gleichgewicht einstellen
	\item Auswirkungen Anharmonizität: thermische Ausdehnung, endlicher Wärmewiderstand, Ultraschallabsorption
\end{itemize}
\textcolor{red}{\bf{\fbox{ACHTUNG:}} Kapitel vorerst übersprungen, Wegener-Skript und Hunklinger starke Abweichungen}
\subsection{Zustandsgleichung und thermische Ausdehnung}
\subsection{Phonon-Phonon-Wechselwirkung}
\subsection{Wärmetransport in dielektrischen Kristallen}
\section{Elektronen im Festkörper}

Gültigkeit der adiabatischen Näherung vorausgesetzt.
\subsection{Freies Elektronengas}
\subsection{Spezifische Wärme}
\subsection{Kollektive Phänomene im Elektronengas}
\subsection{Elektronen im periodischen Potential}
\subsection{Energiebänder}
\section{Elektronische Transporteigenschaften}
\subsection{Bewegungsgleichung und effektive Masse}
\subsection{Ladungstransport}
\subsubsection{Drude-Modell}
\subsubsection{Sommerfeldsche Theorie}
\subsubsection{Boltzmann-Gleichung}
\subsubsection{Elektrischer Ladungstransport}
\subsubsection{Elektronstreuung}
\subsubsection{Temperaturabhängigkeit der elektrischen Leitfähigkeit}
\subsubsection{Eindimensionale Leiter}
\subsubsection{Quantenpunkte}
\subsubsection{Thermische Leitfähigkeit}
\subsubsection{Wiedemann-Franz-Gesetz}
\subsubsection{Fermi-Funktion im stationären Gleichgewicht}
\subsection{Elektronen im Magnetfeld}
In diesem und dem vorhergehenden Kapitel wurde klar, dass Fermi-Flächen eine große Bedeutung für das Verhalten von Festkörpern haben.
\subsubsection{Zyklotronresonanz}
\subsubsection{Landau-Niveaus}
\subsubsection{Zustandsdichte im Magnetfeld}
\subsubsection{De-Haas-van-Alphén-Effekt}
\subsubsection{Hall-Effekt}
\subsubsection{Quanten-Hall-Effekt}
\section{Halbleiter}
\subsection{Intrinsische kristalline Halbleiter}
\subsection{Dotierte kristalline Halbleiter}
\subsection{Inhomogene Halbleiter}
\subsection{Bauteile}
\section{Supraleitung}
\subsection{Phänomenologische Beschreibung}
\subsection{Mikroskopische Beschreibung}
\subsection{Makroskopische Wellenfunktion}
\subsection{Ginzburg-Landau-Theorie und Supraleiter 2. Art}
\section{Magnetismus}
\subsection{Dia- und Paramagnetismus}
\subsection{Ferromagnetismus}
\subsection{Ferri- und Antiferromagnetismus}
\subsection{Spingläser}
\section{Dielektrische und optische Eigenschaften}
\subsection{Elektrische Polarisation von Isolatoren}
\subsection{Optische Eigenschaften freier Ladungsträger}
\end{document}